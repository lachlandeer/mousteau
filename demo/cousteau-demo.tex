\documentclass{beamer}
\usepackage{appendixnumberbeamer}
\usepackage{amssymb,amsmath}
\usetheme{cousteau}

\usepackage[scale=2]{ccicons}

\title{Cousteau: A Beamer Theme of Undersea Discovery and Adventure}
\author{Matthew Blackwell}
\institute{Harvard University}
\date{April 26, 2019}

\begin{document}
\frame{\titlepage}


\begin{frame}{Table of contents}
  \tableofcontents
\end{frame}

\section{Introduction}

\begin{frame}[fragile]{Using Cousteau}
  Cousteau is a Beamer theme based on a nautical minimalism. Some parts of the theme are inspired by or adapted from the Metropolis theme. You can enable Cousteau by using the following commands:
\begin{verbatim}
   \documentclass{beamer}
   \usetheme{cousteau}
\end{verbatim}
  The theme assumes you have the typefaces \textbf{Fira Sans}, \textbf{Fira Mono}, and \textbf{GFS Neohellenic Math} installed. XeTeX is required to compile a slide deck with Cousteau.
\end{frame}


\section{Elements}

\begin{frame}[fragile]{Highlighting words}

\begin{verbatim}
There are several ways to highlight text. One way is to 
\emph{emphasize} what you would like to say or to use 
one of \alert{several} \alertb{accent} \alertc{colors}. 
Another good option is \textbf{bold-face} type. 
\end{verbatim}

  \bigskip
  
  There are several ways to highlight text. One way is to \emph{emphasize} what you would like to say or to use one of \alert{several} \alertb{accent} \alertc{colors}. Another good option is \textbf{bold-face} type. 
  
\end{frame}

\begin{frame}{Fonts}
  \begin{itemize}
    \item Regular
    \item \textit{Italic}
    \item \textsc{Small Caps}
    \item \textbf{Bold}
    \item \textbf{\textit{Bold Italic}}
    \item \textbf{\textsc{Bold Small Caps}}
    \item \texttt{Monospace}
    \item \texttt{\textbf{Monospace Bold}}
  \end{itemize}
  
\end{frame}

\begin{frame}{Lists}
  \begin{columns}
    \column{0.5\textwidth}
    Unordered list:
    \begin{itemize}
    \item RV Calypso
    \item SP-350 Denise
    \item Alcyone
    \end{itemize}

    \column{0.5\textwidth}
    Numbered list: 
    \begin{enumerate}
    \item Un
    \item Deux
    \item Trois
    \end{enumerate}
  \end{columns} 
  
\end{frame}

\begin{frame}{Blocks}

  \begin{block}{Default block}
    Block content.
  \end{block}

  \begin{alertblock}{Alert block}
    Block content.
  \end{alertblock}

  \begin{exampleblock}{Example block}  
    Block content.
  \end{exampleblock}

\end{frame}

\begin{frame}{Math}
  \[f(x) = \frac{1}{2\pi}\exp\left( -\frac{1}{2}x^2 \right)\]
\end{frame}

\begin{frame}{Quotes}
  \begin{quote}
    Four score and seven years ago our fathers brought forth on this continent, a new nation, conceived in Liberty, and dedicated to the proposition that all men are created equal.
  \end{quote}
\end{frame}

\begin{frame}[fragile]{Image frames}
  Sometimes it can be useful to have an image take up the entire frame. Using \texttt{tikz}, Cousteau provides a an easy command to include a full screen image for a slide:
\begin{verbatim}
    \imageframe{kanagawa.jpg}
\end{verbatim}
See the next slide for the result of this command. \texttt{imageframe} will make the image as large as possible without cropping, so there may be some vertical or horizontal space. Crop images to 4:3 for best results. 
\end{frame}

\imageframe{kanagawa.jpg}

 
\section{Conclusion}

\begin{frame}{Summary}
  You can find the source for these demo slides and the entire Cousteau theme at:
  \begin{center}
    \url{http://github.com/mattblackwell/cousteau-theme/}
  \end{center}

  The theme is licensed under a
  \href{http://creativecommons.org/licenses/by-sa/4.0/}{Creative Commons
  Attribution-ShareAlike 4.0 International License}.

  \begin{center}\ccbysa\end{center}

  
\end{frame}

\appendix

\begin{frame}{Appendix slides}

  This is an appendix slide via the \texttt{appendixnumberbeamer} package and the \texttt{\\appendix} command. Note that the frame numbers are reset. 
  
\end{frame}

\end{document}
